\documentclass{article}
\usepackage{fullpage}
\usepackage{amsmath}
\usepackage[english]{babel}

\setlength{\parskip}{.4cm}
\setlength{\baselineskip}{15pt}
\setlength{\parindent}{0cm}

\author{Brett Jurgens}
\date{}

\begin{document}
    Let’s say you’re comparing motif M1 with motif M2.
    
    Each motif is a PWM of length, say, k. So $M11, M12, ..., M1k$
    are the columns of motif M1; and $M21, M22, ..., M2k$ are the columns of M2.
    
    The relative entropy score comparing these two motifs is going to be the sum over all positions, of
    the relative entropy between the corresponding columns.
    
    That is, $RE(M1,M2) = \sum_{i=1}^{k} RE(M1i, M2i)$.
    
    Now the question is how to compute the RE of two columns.
    
    Well, each column (i.e., M1i or M2i) is a probability distribution over the four nucleotides.
    
    And relative entropy is defined over two probability distributions over the same space:
    
    Let $p = (p1, p2, p3, p4)$ and $q = (q1, q2, q3, q4)$ be two probability distributions over the four
    nucleotides. Then RE(p,q) is defined as $\sum_j p_j log (p_j/q_j)$.
    
    Plugging this formula into the computation of RE(M1i, M2i) and summing over all $i = 1, 2, ..., k$,
    you get the RE of the two motifs.
\end{document}